% $Log: abstract.tex,v $
% Revision 1.1  93/05/14  14:56:25  starflt
% Initial revision
% 
% Revision 1.1  90/05/04  10:41:01  lwvanels
% Initial revision
% 
%
%% The text of your abstract and nothing else (other than comments) goes here.
%% It will be single-spaced and the rest of the text that is supposed to go on
%% the abstract page will be generated by the abstractpage environment.  This
%% file should be \input (not \include 'd) from cover.tex.
\section*{\small Abstract}\label{objetivo_sec}
{\renewcommand\baselinestretch{1}\small
  En este trabajo se presenta la arquitectura de hardware para una implementación
  de una convolución bidimensional en una FPGA Xilinix Artix 7.

  Se plantea una solución para cuando no se cuenta con suficiente memoria RAM
  instanciada para almacenar la imagen completa. Se priorizó no solo la
  velocidad de procesamiento sino también un uso eficiente de los recursos y la
  escalabilidad del diseño mediante una estructura modular de forma tal que sea
  posible añadir tantas operaciones de multiplicación/adición en paralelo como
  sea necesario sin requerir modificaciones importantes en el diseño.

  Con la estructura propuesta se logra un reuso dinámico de memoria, con un
  incremento lineal en la utilización de la misma en función del nivel de
  paralelismo. \par}

% El objetivo de la Práctica Profesional Supervisada, en adelante PPS, es lograr
% la inserción laboral del alumno en la realidad profesional del país, en el área
% que mejor responda a sus aspiraciones profesionales e intereses vocacionales,
% con el propósito de fortalecer su formación académica y de establecer un vínculo
% que facilite su ingreso como profesional al mercado de trabajo.
% Este proyecto consta en el diseño e implementación en una FPGA (Field
% Programmable Gate Array) de un sistema destinado al procesamiento digital de
% imágenes vía hardware.
% Fue desarrollado en la institución Fulgor, bajo la supervisión del Dr.Ing. Ariel
% Luis Pola. \par}