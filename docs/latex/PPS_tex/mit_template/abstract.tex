% $Log: abstract.tex,v $
% Revision 1.1  93/05/14  14:56:25  starflt
% Initial revision
% 
% Revision 1.1  90/05/04  10:41:01  lwvanels
% Initial revision
% 
%
%% The text of your abstract and nothing else (other than comments) goes here.
%% It will be single-spaced and the rest of the text that is supposed to go on
%% the abstract page will be generated by the abstractpage environment.  This
%% file should be \input (not \include 'd) from cover.tex.
\section*{\small Objetivo}\label{objetivo_sec}
{\renewcommand\baselinestretch{1}\small
El objetivo de la Práctica Profesional Supervisada, en adelante PPS, es lograr la inserción laboral del alumno en la realidad profesional del país, en el área que mejor responda a sus aspiraciones profesionales e intereses vocacionales, con el propósito de fortalecer su formación académica y de establecer un vínculo que facilite su ingreso como profesional al mercado de trabajo. 
Este proyecto consta en el diseño e implementación en una FPGA (Field Programmable Gate Array) de un sistema destinado al procesamiento digital de imágenes vía hardware. 
Fue desarrollado en la institución Fulgor, bajo la supervisión del Dr.Ing. Ariel
Luis Pola. \par}

\section*{\small Motivación}\label{motiv_sec}
{\renewcommand\baselinestretch{1}\small
El filtrado y procesamiento de imágenes es utilizado en múltiples campos, como la medicina, ingeniería, navegación, aeronáutica, entre otros. En el campo de la inteligencia artificial, área emergente que es de mero interés hoy en día, la convolución en 2D es la operación principal que se realiza durante el feed forward de una CNN (Red neuronal Convolucional). 
Dada su naturaleza, la operación de convolución en 2-D es la más empleada en los algoritmos de procesamiento de imágenes. El paralelismo inherente de algoritmos basados en la convolución es explotado logrando alta performance en estos sistemas que los utilizan.
El uso de Field Programmable Gate Arrays (FPGAs) para implementar este tipo de sistemas, se debe a la ventaja que presentan estos dispositivos en lo que concierne al paralelismo a nivel bit, pixel, vecindad y a nivel tarea, lo incrementa la performance y velocidad de cómputo.
Se presenta una arquitectura para la implementación en hardware de la operación
de convolución 2-D priorizando el uso eficiente de recursos junto con la
escalabilidad del diseño y velocidad de procesamiento. \par}